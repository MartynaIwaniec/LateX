\documentclass{article}
\usepackage[utf8]{inputenc}
\usepackage[english]{babel}
\usepackage[T1]{fontenc}
\usepackage{csquotes}
\linespread{1.5}

\usepackage[
backend=biber,
style=alphabetic,
sorting=ynt
]{biblatex}

\addbibresource{library.bib}

\title{biblatex}
\author{Martyna Iwaniec I MA PJN}
\date{April 2020}

\begin{document}

\maketitle
\begin{abstract}
This \LaTeX{} document was made for learning purposes. It includes a few pages of my Bachelor's thesis.
\end{abstract}

\maketitle

\tableofcontents

\section{First section}
\subsection{}
Although Krashen’s SLA theory is composed of five different hypotheses, “the acquisition-learning hypothesis” \cite{krashen_principles_1982} is often regarded as the most important. It is also most relevant to the current chapter. As the name suggests, the hypothesis proposes that humans possess two different and separate means of obtaining L2: acquisition and learning.
As reported by Krashen, “acquisition” \cite{krashen_principles_1982} is a phenomenon comparable to the one during which people absorb their L1. Naturally, if exposed to MT long enough, children will eventually develop language skills. Additionally, Krashen describes acquisition as an internal mechanism that people normally undergo without realising it, although they recognise that they are applying language to convey messages. This is especially meaningful to Krashen, because he argues that the most essential goal of acquisition is to utilise language in order to exchange information. At the same time, the author argues that rather than being concerned with grammar or syntax, people acquire primarily without knowing. In particular, learners use their new language skills instinctively and this is the most natural way; hence, they are able to identify what is more likely to be accurate and what seems incorrect. Thus, people acquire a language without feeling like they are studying. As stated earlier, the only goal of such a process, therefore, is to convey messages efficiently \cite{krashen_principles_1982}.
In contrast to acquisition, Krashen describes “learning” \cite{krashen_language_1989} as the entirety of information about L2 that people realise they have gathered. To put it simply, it is the features of a language that learners are capable of discussing, for example, syntax. Thereupon, it can be deduced that such a mechanism occurs most often in FL classrooms. Apart from this, Krashen claims that learning provides people with a single competence: while producing utterances, learners are able to pause, reassess and alter the sentences in accordance with the governing principles of a language. Nevertheless, the scholar admits that this is extremely challenging to achieve due to the fact that it requires language users to concentrate on structure and accuracy, and at the same time be aware of the rules \cite{krashen_language_1989}.
	In other words, Krashen believes that acquisition is simply “picking-up a language” \cite{krashen_principles_1982}, whereas learning can be defined as “knowing about a language”. Subsequently, the acquisition-learning hypothesis contradicts the popular notion that only grown people can learn, while the young always acquire. In fact, it is usually much harder to attain such status after childhood. Specifically, the aforementioned theory assumes that people can still acquire a language even after they go through adolescence, although, it does not promise that they can reach “native-like level” \cite{krashen_principles_1982}. 
Along with the acquisition-learning hypothesis, Krashen’s theory of L2 acquisition incorporates “the natural order hypothesis” \cite{krashen_principles_1982}. It was developed as a result of the discovery that “the acquisition of grammatical structures proceeds in a predictable order” \cite{krashen_principles_1982}. Noticeably, the sequence in which a tongue is acquired in L1 seems to be different to the one appearing in L2. Nonetheless, the author reports that obvious correspondences between them has been observed \cite{krashen_principles_1982}.
After the acquisition-learning hypothesis, Krashen describes “the Monitor hypothesis” \cite{krashen_principles_1982} which assumes “that acquisition and learning are used in very specific ways”. Albeit the former is generally responsible for producing pieces of language, the latter acts as a “Monitor, or editor” \cite{krashen_principles_1982}. To put it simply, learning is involved solely in revising the output created by acquisition.
Consequently, as stated by Krashen, “the input hypothesis” (\cite{krashen_principles_1982} is especially important, because it tries to answer a crucial question: “How do we acquire language?” \cite{krashen_principles_1982}. For that reason, the scholar writes about it extensively, however, the present thesis will include only the most essential information on the subject. Initially, Krashen stresses the fact that merely using the target language does not amount to acquiring it. Then, the researcher explains that acquisition occurs when there is some information included that comes from outside a person’s present stage of language proficiency. In fact, the author announces that SLA ensues due to “context or extra-linguistic information” \cite{krashen_principles_1982}.
Finally, “the Affective Filter hypothesis” \cite{krashen_principles_1982} maintains that a particular state of mind can negatively affect SLA. For instance, seemingly harmless emotions such as shyness, nervousness or boredom may have as great of an influence on someone’s performance as lack of ambition or insufficient belief in one’s abilities. Besides affecting the input, poor mood can also cause “a high or strong Affective Filter” \cite{krashen_principles_1982}. In this case, pieces of language would not be acquired, although the information would be comprehensive to the learners. Therefore, according to Krashen, “a lower or weaker filter” \cite{krashen_principles_1982} is much more effective and it is most beneficial to the learners. 
\subsection{}
 Rod Ellis’ research
 Rod Ellis admits that the concept of L2 acquisition can be arduous to describe. Most importantly, unlike Krashen, Ellis does not separate learning from acquisition in any way. According to Ellis, SLA takes place when people learn either with a sense of curiosity, such as during lessons, or subconsciously, which is noticeable beyond the school setting. Meanwhile, Krashen’s definition maintains that people can undergo both acquisition and learning, although, the phenomena occur on specific and distinct occasions. Ellis, however, explains that various theorists hold different opinions on how to detect and calculate acquisition, so the term may represent numerous ideas. It is worth mentioning that for Ellis, concepts of “naturalistic and instructed” \cite{ellis_study_1994} SLA are especially noteworthy. In short, the former takes place during contact with other language users in everyday life, while the latter occurs while receiving formal education, usually with a teacher present \cite{ellis_study_1994}.
Because Ellis believes that acquisition occurs in a certain order, it can be said that his assumption bares resemblance to Krashen’s natural order hypothesis. Further, Ellis asserts that while people incorporate “chunks of language structure” \cite{ellis_study_1994}, they “acquire rules” \cite{ellis_study_1994} as well. According to Ellis, it can be concluded that the aim of SLA is to analyse and to define semantic and expressive abilities of the acquirer. What is more, the scholar believes that the phenomenon is assumed to have transpired if a learner is able to utilise L2 correctly or does it for the very first time \cite{ellis_study_1994}.
Noteworthy, Ellis specifies that the term SLA does not apply only to L2 that someone learns, but that it might also be used to describe any language besides MT. Nonetheless, lack of differentiation between the names of acquired systems can lead to confusion when someone knows more than two tongues, so Ellis proposes the expression “additional language” instead \cite{ellis_study_1994}. Subsequently, the author explains that “a distinction between second and foreign language is sometimes made” \cite{ellis_study_1994}. According to Ellis, the primary function of L2 is to allow people whose L1 is another tongue to exchange messages in social situations. Conversely, FL may be usually found in schools, not in its natural environment. In light of this, it is clear that not only are L2 and FL unalike in the methods employed in learning, but they differ in the material that is acquired. Nevertheless, the scholar chooses to ignore the differences in those two kinds of learning situations and to call them both SLA, no matter how a person obtains the language \cite{ellis_study_1994}.
\subsection{}
Janusz Arabski’s research
Furthermore, Janusz Arabski claims that learning is a phenomenon distinct from acquisition. Specifically, he asserts that acquisition concerns either MT or a situation during which a child obtains L2 intuitively, in the language’s natural setting. Accordingly, the author emphasizes that acquisition must happen without any teachers and at the same time, the acquirer cannot be aware of undergoing any educational process. Moreover, learning occurs in an educational environment, usually at school, and affects predominantly the youth and adults \cite{arabski_przyswajanie_1996},. In particular, Arabski explains that the reason for such situation is the fact that children are marked with greater brain plasticity, which favours speech acquisition. Unfortunately, this neuroplasticity diminishes with age \cite{arabski_przyswajanie_1996},.
In addition, Arabski distinguishes between L2 and FL. As declared by the scholar, L2 is concerned with learning or acquiring a language in its natural setting, while FL learning belongs in an artificial environment. For instance, L2 can be found in a country where it is used on a daily basis. In particular, if a person is surrounded by a new tongue in a foreign country and wants to communicate with its citizens, he or she will have to acquire the L2. Further, the motivation for acquisition is strong and the aim is clear, since the quality of somebody’s life often depends on it. Thus, contact with L2 is usually frequent and consistent. In contrast, FL can be found in classrooms, therefore, outside of the community that uses it. At the same time, because the purpose of learning FL seems distant and not very clear to the students, the motivation for it is weaker than for the acquisition of L2. In general, the most compelling distinction between the two is the level of motivation and the frequency of use of the tongue. Although the majority of Arabski’s writings concern L2 rather the than FL, he admits that the two topics are similar, and they vary mostly in the intensity of language.
\section{Second Section}
\subsection{}
Communication inside and outside the classroom
It is imperative to remember that the language used in real-life situations is noticeably distinct from the one in a school setting. When considering the main differences between natural and artificial language, Komorowska argues that the biggest factor is the unpredictability of the former. Subsequently, when pupils encounter L2 outside the classroom, they often have problems with understanding it or with producing utterances because of the unpredictability of real-life situations. For this reason, the authenticity of L2 is vital during FL lessons. Moreover, redundancy is common in everyday language, because people tend to make communication economical and to simplify it. Whereas, in the classroom, learners are provided with all elements and do not have to guess anything. Unlike during L2 lessons, when people use language in natural situations, they communicate with facial expressions, gestures, the tone of their voice and even with their eyes. Thus, it should come as no surprise that in everyday life a large portion of information is conveyed nonverbally \cite{komorowska_podstawy_1993}. 
In addition, Komorowska states that an authentic situation includes some form of information gap, which means that there is a disparity in the levels of information for the interlocutors. When a teacher asks the learners a question, they usually already know the answer, because in the school setting, it is the form that matters and not the content. Another distinction is the fact that in the classroom, the language skills are practised one at a time, so it is not too difficult for the pupils. Normally, real life conversations require the use of several skills simultaneously. Lastly, authentic language is very diversified, and it depends largely on who are the speaker and the addressee, what is the context, how formal is the situation and many other factors \cite{komorowska_podstawy_1993}.


\medskip
\clearpage

\printbibliography[
heading=bibintoc,
title={Whole bibliography}
]

\end{document}
